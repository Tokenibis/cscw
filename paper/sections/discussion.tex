I now discuss the key insights and takeaways from the study.
The objective of this study was to explore the potential of UBP as a mechanism for computing philanthropic preferences.
To this end, a thorough analysis of the data revealed coherent structures and patterns that align with intuitive explanations.
The analysis is almost entirely automated except for one step: the translation of keywords from organizational clusters into named causes.
The single most notable takeaway is the delineation of organizations that achieve comparatively high intensity and organizations that achieve comparatively high connectivity.

Having generated results in for this small-scale pilot, a natural question raises: what are the concerns and opportunities associated with a larger-scale deployment?
Drawing on my longitudinal experience with the project, I recommend that large-scale deployments consider the following risks:

\begin{itemize}
  \item \textbf{Fraud}: Deployments should impose proper identity checks to make sure that individuals cannot falsify duplicate accounts to increase their access to funds.
  \item \textbf{Exclusivity}: Deployments should ensure that organizations do not form to exclusively benefit their members without providing a sufficiently distributed social good.
  \item \textbf{Reciprocity}: Deployments should ensure that organizations do not offer material benefits in exchange for donations, thus compromising the public benefit.
\end{itemize}

These risks---which mirror considerations in the existing third sector---will require nontrivial mitigations.
On the other hand, it may be the case that additional scale alleviates some of the issues with the small-scale experiment.
Chief among these are the uneven flow of funds in the Albuquerque pilot project\footnote{If the Albuquerque pilot organizations formed a country, then the Gini Coefficient among organizations would be 0.56. According to the Central Intelligence Agency World Factbook (\url{https://www.cia.gov/the-world-factbook/field/gini-index-coefficient-distribution-of-family-income/country-comparison/}), this score would make it the third most unequal country in the world after Namibia.}.
In comparison, a large-scale deployment might be more decentralized along the following dimensions:

\begin{itemize}

  \item \textbf{Organizations}: A large-scale deployment might feature a flatter distribution of fundraising success at the top.
    The organizations that dominated activity in the Albuquerque pilot had relatively small budgets and support bases outside of the platform.
    Consequently, it seems plausible that there is a natural upper limit to the amount of support they could maintain on a larger UBP deployment, thus leaving more available funding for other organizations.

  \item \textbf{People}: A large-scale deployment might be more attractive to a wider diversity of users.
Anecdotally, casual users who do not have strong organizational affiliations have described the value of the platform as a way to learn about new organizations.
In principle, a large deployment with greater network effects would appeal to more users of this type.

\end{itemize}

Finally, the designs of the Albuquerque pilot and this study can be extended beyond the scope of local nonprofits.
For instance, focusing on larger, internationally-focused NGOs would reduce the risk of membership-based biases while elucidating geopolitical sentiments over time.
As another example, focusing on political campaign donations would increase legally-mandated transparency, while elucidating user preferences for more contested forms of social impact.  
These domains, while different in substance, share the same structural need for a systematic mechanism to fund public goods.
Through the principles of simple design and data-driven analysis, this research introduces such a mechanism: a universal means to convert democratic preferences into social profit.
