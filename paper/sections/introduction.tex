Amid the ubiquitous reach of state and corporate power, some see the third sector---the subset of the economy that includes international non-government organizations (NGOs), domestic nonprofits, and volunteer communities---as the most promising domain for true social progress.
The notional appeal is clear: a way to fund public goods that combines the prosocial values of public government with the productive efficiency of private enterprise~\cite{etzioni1973third}.
This promise has, in part, motivated a wealth of academic research into the structural nature of philanthropic giving.
Examples of relevant work include methods to correlate donations with social networks~\cite{apinunmahakul2008social}, geographic proximity~\cite{chapman2022give}, and social media activity~\cite{korolov2016predicting}~\cite{saxton2014social}.
However, serious concerns persist over the imbalanced power dynamics of this inherently plutocratic institution~\cite{maclean2021elite}~\cite{reich2020just}.
The resulting ideological tension between charity and egalitarianism has driven interest in participatory philanthropy, a collection of approaches that emphasize more community input into the decision-making process~\cite{meyer2021walking}~\cite{hauger2023nothing}~\cite{bhati2020literature}.

One concrete approach, called \emph{Universal Basic Philanthropy} (UBP), is the subject of this research.
Under UBP, all members in some well-defined set of individuals periodically receive a dedicated stipend that they can use to donate to approved nonprofit organizations.
UBP differs from conventional philanthropy in the equitable distribution of funding power.
It differs from other participatory philanthropy models, which often involve collective voting procedures, in its prioritization of individual decision-making.
In general, UBP offers an interesting setting for several potential research directions, among them: psychological considerations of personal well-being~\cite{anik2009feeling}, sociological considerations of class dominance~\cite{silver2007disentangling}, and economic considerations of local knowledge transfer~\cite{wandel2014nonprofit}.
However, research efforts in these directions---which would require supplementary theory and data---are best left for more substantial future work.
Instead, this preliminary study seeks to break ground on a more tractable, self-contained topic: UBP as a social mechanism for computing revealed philanthropic preferences.
In principle, there are two reasons why this direction is compelling. 
Since individuals do not have to spend their own money, UBP donations are distinctively:

\begin{itemize}
  \item \textbf{Representative}: The magnitude of donations does not depend on a donor's private wealth. Consequently, influence in the UBP activity is theoretically less impacted by socioeconomic status~\cite{james2007nature}.
  \item \textbf{Deliberate}: The calculus of decision-making does not depend on personal financial sacrifice. Consequently, activity in UBP is theoretically less impacted by spontaneous impulses~\cite{bennett2009impulsive}.
\end{itemize}

In this study, I assess the first known instance of a live UBP ecosystem through the lens of revealed preference.
The target platform, a real-world pilot project set in the city of Albuquerque, New Mexico, provides public access to a complete digital record of activity.
Using this resource, I describe an end-to-end process for extracting high-level insights from low-level data.
I find that inline with focus theory, the type of social cause that binds members in an organization helps determine its network centrality~\cite{feld1981focused}.
Throughout the assessment, I augment the quantitative data with qualitative insights and speculations from my personal experience with the project.\footnote{I am a founding member and sitting board member of the operating organization. However, as a volunteer contributor, I do not have any financial conflicts of interest to disclose.}
The purpose of this commentary is only to help ground intuitions and provide context for the chosen methodologies.
The primary intended contribution of this work is a computational description of the process itself.~\footnote{Data and code are publically available at \url{https://github.com/Tokenibis/cscw}}
By documenting the design and analysis of a real-world experiment, my goal is to establish a coherent basis for future efforts.
